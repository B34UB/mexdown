\documentclass[12pt]{article}

\usepackage{url}
\usepackage{a4wide}
\usepackage{moreverb}
\usepackage{rotating}

\usepackage{tikz-er2}


\newcommand{\demo}[1]{
  \bigskip
  \begin{minipage}{\linewidth}
      \begin{center}
          \input{#1}
      \end{center}
      \begin{center}
          \verbatiminput{#1}
      \end{center}
  \end{minipage}
}


\begin{document}

\title{The \emph{Tikz-er2} Package for\\Drawing Entity-Relationship Diagrams}
\author{P\'avel Calado\footnote{Contact information can be found at \protect\url{http://web.ist.utl.pt/pavel.calado/}}}
\maketitle

\tikzstyle{every edge} = [link]

\tableofcontents

\begin{abstract}
    This document describes \emph{Tikz-er2}, a \LaTeX\ package for drawing
    Entity-Relationship diagrams. \emph{Tikz-er2} uses Tikz/PGF for graphics
    and is very easy to use. In the following, you can find instructions on how
    to draw each individual E-R element and an example of a full E-R diagram.
\end{abstract}

\section{The \emph{Tikz-er2} Package}
\label{sec:emphtikz-er2-package}

\emph{Tikz-er2} is a \LaTeX\ package that uses the
Tikz/PGF\footnote{\url{http://sourceforge.net/projects/pgf/}} library to draw
Entity-Relationship diagrams.  It is, in practice, a set of Tikz styles, which
you can use together with your usual Tikz instructions to draw the diagrams.

Tikz already has an E-R diagram library.  However, I felt it was very
incomplete, lacking some of the features I usually require when designing
databases. Thus, I implemented \emph{Tikz-er2}, which uses the E-R notation
described in the book ``Database System Concepts'', by Silberchatz et
al.\footnote{Abraham Silberschatz, Henry F. Korth, and S. Sudarshan: ``Database
System Concepts''. 5th Edition, McGraw-Hill, 2005.}

This package, as everything else in the Tikz library, is very easy to use. In
the following sections, you will find a full set of examples of how to use it to
create an E-R diagram.  

I should note that, although I believe these examples are enough to draw any
(moderately complex) diagram, I strongly recommend reading the Tikz/PGF
manual. Besides being useful if you need to draw something that I did not
remember to exemplify here, you will be amazed with what Tikz/PGF can do for
your \LaTeX\ graphics.


\section{Drawing E-R Diagrams}
\label{sec:drawing-e-r}

To use the \emph{Tikz-er2} package you should include the command
\begin{center}
    \ttfamily
    \verb+\usepackage{tikz-er2}+
\end{center}
in the preamble of your document. This section shows how to draw each individual
element in an E-R diagram. Each figure is shown together with the code used to
generate it.


\subsection{Entity Sets}
\label{sec:entities}

Entity sets are represented using the \texttt{entity} style. The following code
draws an entity set element. An example, with the respective \LaTeX\ code is
shown in Figure~\ref{fig:entity}.

\begin{figure}[htb!]
    \centering
    \demo{entity}
    \caption{Entity set.}
\label{fig:entity}
\end{figure}

You can change the style of the elements by redefining the \texttt{every
  entity} style, as shown in Figure~\ref{fig:entity-blue}.

\begin{figure}[htb!]
    \centering
    \demo{entity-blue}
    \caption{Entity set, with a different style.}
\label{fig:entity-blue}
\end{figure}


\subsection{Attributes}
\label{sec:attributes}

You can draw all types of attributes, as seen in Figure~\ref{fig:attribute}.

\tikzstyle{every entity} = [fill=blue!30]
\tikzstyle{every attribute} = [fill=yellow!20]
\begin{figure}[htb!]
    \centering
    \demo{attribute}
    \caption{Attribute.}
\label{fig:attribute}
\end{figure}

Regular attributes are drawn using the \texttt{attribute} style. There is a
style \texttt{multi attribute} for multi-valued attributes and a style
\texttt{derived attribute} for derived attributes. Key attributes are draw by
using the command \verb+\key+ on its label, to get underlined text.

As for entity sets, you can change the style of the attributes by redefining the
\texttt{every attribute} style.


\subsection{Relationship Sets}
\label{sec:relationships}

\tikzstyle{every relationship} = [fill=red!20]

There is a \texttt{relationship} style for relationship sets. This example
shows a \emph{many-to-many} relationship. In this case, we changed the style of
all relationship sets by redefining the \texttt{every relationship} style.

\begin{figure}[htb!]
    \centering
    \demo{relationshipM-M}
    \caption{Relationship set.}
\label{fig:relationship}
\end{figure}

There is also a style for the links connecting relationship sets to entity sets
(and to attribute sets): the \texttt{link} style. In
Figure~\ref{fig:relationship}, instead of using the \texttt{link} style on
every edge, we simply redefined the \texttt{every edge} style to use it.

To draw a \emph{one-to-many} relationship, or a one to one relationship, you
just need to draw an arrowhead on the respective edge. To draw a total
participation you can use the style \texttt{total} on the corresponding edge. An
example of both cases is shown in Figure~\ref{fig:relationship1-M}.

\begin{figure}[htb!]
    \centering
    \demo{relationship1-M}
    \caption{One-to-many relationship set, with a total participation.}
\label{fig:relationship1-M}
\end{figure}

You can also use an alternative notation, were you specify the cardinality of
the relationship as a number on the links. In this case, you just need to add a
label to the respective link, as seen in Figure~\ref{fig:relationship-card}.

\begin{figure}[htb!]
    \centering
    \demo{relationship-card}
    \caption{Alternative notation for relationship set cardinality.}
\label{fig:relationship-card}
\end{figure}

The same method can be used to add roles to the participant entities. An
example is shown in Figure~\ref{fig:roles}.

\begin{figure}[htb!]
    \centering
    \demo{roles}
    \caption{Roles in an E-R diagram.}
\label{fig:roles}
\end{figure}


\subsection{Weak Entity Sets}
\label{sec:weak-entities}

The style for weak entities is \texttt{weak entity}, shown in
Figure~\ref{fig:weak}. The identifying relationship uses style
\texttt{ident relationship}. The \verb+\discriminator+ command is used to
draw a dashed underline, to represent the discriminator attribute.

\begin{figure}[htb!]
    \centering
    \demo{weak}
    \caption{Weak entity set.}
\label{fig:weak}
\end{figure}

To redefine the style for all weak entities you can use the \texttt{every weak
  entity} style.

\subsection{Specialization/Generalization}
\label{sec:spec}

A specialization/generalization is drawn using the \texttt{isa} style,
exemplified in Figure~\ref{fig:isa}.

\tikzstyle{every isa} = [fill=green!20]
\begin{figure}[htb!]
    \centering
    \demo{isa}
    \caption{Specialization/generalization.}
\label{fig:isa}
\end{figure}

To draw a total specialization/generalization, we can use the \texttt{total}
style on the corresponding link.  To add constraints, such as
\emph{disjoint/overlapping}, we can add a label to the corresponding link.

As before, you can also redefine the \texttt{every isa} style, to change the
appearance of all specialization/generalization symbols.

\subsection{Aggregation}
\label{sec:aggregation}

Finally, you can draw aggregations by simply joining two relationship sets, as
shown in Figure~\ref{fig:aggregation}.

\begin{figure}[htb!]
    \centering
    \demo{aggregation}
    \caption{Aggregation.}
    \label{fig:aggregation}
\end{figure}

The rectangle around the aggregation can be drawn using the functionalities
provided by the Tikz \texttt{fit} library, already included by the
\emph{Tikz-er2} package.


\section{A Full Example}
\label{sec:full-example}

In Figure~\ref{fig:cars} is a an example of a simple complete E-R diagram. It
represents a database for a used car store. It was used as an exercise in the
databases course I teach at IST\footnote{http://www.ist.utl.pt}. The code for
this example can be found in Appendix~\ref{sec:tikz-code-full}.

\begin{sidewaysfigure}[p!]
    \usetikzlibrary{positioning}
\usetikzlibrary{shadows}

\tikzstyle{every entity} = [top color=white, bottom color=blue!30, 
                            draw=blue!50!black!100, drop shadow]
\tikzstyle{every weak entity} = [drop shadow={shadow xshift=.7ex, 
                                              shadow yshift=-.7ex}]
\tikzstyle{every attribute} = [top color=white, bottom color=yellow!20, 
                               draw=yellow, node distance=1cm, drop shadow]
\tikzstyle{every relationship} = [top color=white, bottom color=red!20, 
                                  draw=red!50!black!100, drop shadow]
\tikzstyle{every isa} = [top color=white, bottom color=green!20, 
                         draw=green!50!black!100, drop shadow]

\begin{tikzpicture}[node distance=1.5cm, every edge/.style={link}]
    \node[entity] (emp) {Employee};
    \node[attribute] (ename) [above=of emp] {Name} edge (emp);
    \node[attribute] (enum) [above right=of emp] {\key{Number}} edge (emp);

    \node[isa] (isa) [below=1cm of emp] {ISA} edge (emp);

    \node[entity] (mec) [below left=1cm of isa] {Mechanic} edge (isa);
    \node[entity] (sal) [below right=1cm of isa] {Salesman} edge (isa);

    \node[relationship] (does) [left=of mec] {Does} edge (mec);

    \node[weak entity] (rep) [below=of does] {RepairJob} edge (does);
    \node[attribute] (rnum) [left=of rep] {\discriminator{Number}} edge (rep);
    \node[attribute] (desc) [above left=of rep] {Description} edge (rep);
    \node[attribute] (cost) [below left=of rep] {Cost} edge (rep);
    \node[attribute] (mat) [left=0.5cm of cost] {Parts} edge (cost);
    \node[attribute] (work) [below left=0.5cm of cost] {Work} edge (cost);

    \node[ident relationship] (reps) [below=of rep] {Repairs} edge [total] (rep);

    \node[entity] (car) [right=of reps] {Car} edge [<-] (reps);
    \node[attribute] (lic) [above=of car] {\key{License}} edge (car);
    \node[attribute] (mod) [below=of car] {Model} edge (car);
    \node[attribute] (year) [below right=of car] {Year} edge (car);
    \node[attribute] (manu) [below left=1.5cm of car] {Manufacturer} edge (car);
    
    \node[relationship] (buy) [below=of sal] {Buys};
    \node[attribute] (pri) [above left=of buy] {Price} edge (buy);
    \node[attribute] (sdate) [left=of buy] {Date} edge (buy);
    \node[attribute] (bval) [below left=of buy] {Value} edge (buy);

    \node[relationship] (sel) [right=of buy] {Sells};
    \node[attribute] (sdate) [above right=of sel] {Date} edge (sel);
    \node[derived attribute] (sval) [right=of sel] {Value} edge (sel);
    \node[attribute] (com) [below right=of sel] {Comission} edge (sel);
    
    \draw[link] (car.10) -| (buy) (buy) edge (sal);
    \draw[link] (car.-10) -| (sel) (sel) |- (sal);

    \node[entity] (cli) [below right=0.5cm and 3.7cm of car] {Client};
    \node[attribute] (cid) [right=of cli] {\key{ID}} edge (cli);
    \node[attribute] (cname) [below left=of cli] {Name} edge (cli);
    \node[multi attribute] (cphone) [below right=of cli] {Phone} edge (cli);
    \node[attribute] (cadd) [below=of cli] {Address} edge (cli);

    \draw[link] (cli.70) |- node [pos=0.05, auto, swap] {buyer} (sel);
    \draw[link] (cli.110) |- node [pos=0.05, auto] {seller} (buy);
\end{tikzpicture}

    \caption{E-R diagram for a used car store database.}
\label{fig:cars}
\end{sidewaysfigure}


\section{Other Information}
\label{sec:other-information}

This package can be obtained at
\url{http://www.assembla.com/spaces/tikz-er2}. You can use it
free of charge for whatever you want, in whatever way you need.

The only thing I ask is that you let me know if you make any modifications to
the package. If they can be useful to everyone, I'll be glad to insert them in
the next version, giving the proper credit to the author, of course.

\newpage

\appendix

\section{Tikz Code for the Full Example}
\label{sec:tikz-code-full}

\verbatiminput{cars}    


\end{document}
